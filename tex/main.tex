\documentclass[a4paper,12pt]{jsbook}

% パッケージのインポート
\usepackage{graphicx}  % 画像の挿入
\usepackage{amsmath}   % 数式のサポート
\usepackage{hyperref}  % ハイパーリンク
\usepackage{geometry}  % ページレイアウトの調整
\geometry{margin=1in}  % マージンの設定

% hyperrefの設定
\hypersetup{
    colorlinks=true,
    linkcolor=blue,
    filecolor=magenta,
    urlcolor=cyan,
    pdftitle={交通データ解析報告書},
    bookmarks=true,
    pdfpagemode=FullScreen,
}

\title{令和4年度松山都市圏パーソントリップ調査\\分析レポート
\vspace{10cm}}
\author{ \Large 増田 慧樹 \\
東京大学 工学系研究科 社会基盤学専攻
}
\date{}

\begin{document}

\maketitle

\begin{abstract}
令和4年度に松山都市圏(松山市、伊予市、東温市、砥部町、松前町)で行われたパーソントリップ調査の分析結果から、松山都市圏の都市計画・交通計画の現状と課題を整理する。
\end{abstract}

\tableofcontents

\chapter{はじめに}
都市部と地方部の人口の極端な偏り、少子高齢化による地域の衰退、公共交通のサービス低下、気候変動による災害リスクの増加など、日本の地方都市では、多くの課題が顕在化している。
私たちは、高度成長期の終わりから散々危惧されてきた上記の課題に対して、未だ有効な解決策を打てているとは言えない。
都市計画分野では、コンパクトシティの理念のもと、立地適正化計画が各自治体で策定され、公共交通の利便性の高い地域に居住と都市機能を集約し、効率的な都市運営が目指されている。
しかし、実態は、資産価格の下落を懸念する既存の住民との調整は困難であるため、集約ゾーンの保守的な決定と、強制力の低い政策に終始しており、目指す都市像が達成できるかは疑問である。
交通計画においても、公共交通の利用促進や、歩行者まちづくり、自動運転・デマンド交通システムの導入が、さまざまな自治体で検討されているが、依然として自動車に過度に依存した社会から抜け出せてはいない。

重要なことは、これらの政策は目的として語られるべきではなく、手段のひとつに過ぎないということだ。
全ての地域に適用できる特効薬などあるはずもないから、政策ありきではなく、地域で人がどのように生活し、何に困っているのか、あるいは今後困ると予想されるのかを、まずは地道に見つめ直さなくてはならない。
その根拠の一つとなるのが\textbf{データ}である。
松山都市圏パーソントリップ調査では、松山都市圏に住む19000人以上の生活が記録されている。
子育て世帯、高齢者、高校生、一人暮らしの会社員、郊外に住む人、街中に住む人、さまざまな人の日々の活動と移動が収められている。
ここに記録されていることを丁寧に、網羅的に解釈することで、冒頭の表層的な課題の根本にある、ひとりひとりの生き方が深い理解として得られるだろう。
そうして見出された地域の実情をひとつひとつ汲み取って、将来に向けて計画していくことでしか、地域を良くすることはできないのではないか。

しかし、データは、データ単体としては力を持たない。
膨大なデータを前に、何に着目するのか、どのような分析を行うかは、分析者の裁量であり、分析者の価値観が反映される。
そのような価値観は、地域を実際に歩き、観察し、語り合った経験からしか生まれない。
また、一人の分析者が経験し、感じ取れることには限界がある。
データが地域を良くするためにあるのならば、どのような分析を行うか・何に着目して何を示すかという分析のプロセス自体に、住民が関わることが必要だと考える。
本レポートでは、私が現時点での経験から着想しうるデータの分析を列挙するが、地域に生きる人がそれぞれの問題意識をもとに、本レポートの分析の視点をアップデートしていくことを望む。

また、そのためのツールとして、データ可視化ダッシュボードを開発している。
本文書やダッシュボードを地域の人が囲みながら、自分とは違う人の生活や困難に目を向けることで、まちづくり活動や行政の計画に地域を生きる人の視点が反映され、誰もが生きやすい地域が実現されることを目指していく。


\chapter{データの概要}
\begin{itemize}
  \item 世帯票
  \item 世帯個人票
  \item 個人票
  \item 付帯調査票
\end{itemize}


\chapter{松山都市圏全体の分析}


\chapter{各地域の分析}
解析結果を図や表を用いて示します。

\chapter{都市圏の課題と政策の提案}
結果についての考察を行います。

\chapter{おわりに}
報告書の結論をまとめます。

\begin{thebibliography}{9}
\bibitem{example2020}
著者名,
論文のタイトル,
ジャーナル名,
2020.
\end{thebibliography}

\end{document}
